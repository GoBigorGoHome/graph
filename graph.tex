\documentclass[fontset=ubuntu]{ctexbook}

\begin{document}
    \chapter{图论高级算法}
    \section{最大流}
    最大流算法分成两大类:增广路(augmengting path)算法与预流推进(preflow-push)算法。
    这一节介绍的三个算法,都属于增广路算法。
    下面给出几个定义。
    \subsubsection*{流网络}
    \emph{流网络}$G=(V,E,c,s,t)$是一个有向图,$V$、$E$是其点集与边集,点和边的数目分别记作$n$、$m$。 每条边\footnote{有向图的边也称作弧} $(u,v)\in E$ 都有一容量$c(u,v)$。$s$和$t$是网络中的两个特殊点,称作源点和汇点。
    
    \subsection{EK算法}
    \subsection{Dinic算法}
    \subsection{ISAP算法}
    \subsection{网络流的建图}
    \section{费用流}
    \section{二分图}
    \subsection{最大流和二分图}
    \subsection{匈牙利算法}
    \subsection{二分图模型应用}
    \section{图的连通}
    \subsection{强连通-Tarjan算法}
    \subsection{双连通}
    \subsection{2-SAT问题}
\end{document}