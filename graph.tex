\documentclass[fontset=ubuntu]{ctexbook}
\usepackage{amsmath}
\usepackage{amsthm} % proof
\usepackage{amssymb} % \mathbb
% \usepackage{amstext} % \text
\newtheorem{theorem}{定理}
\begin{document}
    \chapter{图论高级算法}
    \section{最大流}
    最大流算法分成两大类:增广路(augmengting path)算法与预流推进(preflow-push)算法。
    这一节介绍的三个算法,都属于增广路算法。
    下面给出几个定义。
    \subsubsection*{流网络}
    网络流的研究对象是流网络。\emph{流网络}$G=(V,E,c,s,t)$是一个有向图,$V$、$E$是其点集与边集,点和边的数目分别记作$n$、$m$。$c$是容量函数,每条边($(u,v)\in E$都有一容量$c(u,v)\in \mathbb{N}$。$s$和$t$是网络中的两个特殊点,称作源点和汇点。为简便计,流网络简称「网络」或「图」,简记作$G=(V,E)$。

    自环在网络中无意义,我们规定图$G$中不含自环。%允许网络中有重边。
    下文在论述、证明关于网络流的原理、性质或定理时,为了表示上的方便,我们对流网络做出两条限定:
    \begin{enumerate}
        \item 图中不存在重边;\label{Restrict:1}
        \item 图中不存在反向边,即若$(u,v)\in E$,则$(v,u)\notin E$。\label{Restrict:2}
    \end{enumerate}
    这两条限定都不妨碍一般性。我们可以通过将容量相加将重边合为一条边,反向边可以通过新增一个节点来消除。请读者注意,上文所谓「表示上的方便」是指一条边可以通过两个端点唯一确定。下文我们要介绍的算法和代码可以处理含有重边或反向边的图,这两条限定都不是根本性的,仅仅是为了方便表述而已。

    \subsubsection*{流}
    流是满足下述两个性质的实值函数$f\colon V\times V\to\mathbb{R}$:
    \begin{description}
        \item[容量限制:]对任意$u,v \in V$,有 $0\le f(u,v)\le c(u,v)$。
        \item[流守恒:]对任意$u\in V-\{s,t\}$,有$\sum\limits_{v\in V}f(v,u)= \sum\limits_{v\in V}f(u,v)$
    \end{description}
    $f(u,v)$即边$(u,v)$上的流量,若$(u,v)\notin E$,$f(u,v) = 0$。
    从源点$s$到汇点$t$的总流量称作流$f$的值,记作$|f|$,不难得出
    $$|f| = \sum_{v\in V}f(s,v) - \sum_{v\in V}f(v,s)\text{,}$$
    最大流问题即求给定的网络$G$中的一个值最大的流。

    \subsection{增广路方法}
    增广路方法是求解最大流问题的一种方法。本章要介绍的三个最大流算法都是基于增广路方法的。增广路算法涉及三个重要概念:残余网络,增广路,割。
    \subsubsection*{残量网络}
    给定流网络$G=(V,E)$和$G$上的一个流$f$,残量网络$G_f$是由$G$和$f$所导出的一个网络。首先定义残余容量$c_f$:
    \[
    c_f(u,v) =
    \begin{cases}
        c(u,v) - f(u,v) & \text{若$(u,v)\in E$,}\\
        f(v,u) & \text{若$(v,u)\in E$,} \\
        0 & \text{其他情况.}
    \end{cases}
    \]
    这里需要指出我们提出限制\ref{Restrict:2} 的用意。$(u,v)\in E$和$(v,u)\in E$同时成立会给$c_f$的定义带来形式上的不便。
    残量网络$G_f$定义为$G_f = (V, E_f)$,其中$E_f = \{(u,v)\in V\times V\colon c_f(u,v)>0\}$。除了可能含有反向边,残量网络也符合流网络的定义;而我们已经指出「不含反向边」并非根本性的要求,我们可以用残余容量$c_f$类似地定义残量网络上的流,称作\emph{残量流}。

    我们考虑残量流的原因在于,借助残量流$f'$,可以将原网络$G$上的流$f$修改成一个值更大的流$f\uparrow f'$。用$f'$ 增广 $f$,这正是「增广」二字含义所在。增广方法为:
    \[
    (f\uparrow f')(u,v) =\begin{cases}
    f(u,v) + f'(u,v) - f'(v,u) & \text{若$(u,v)\in E$,} \\
    0 & \text{其他情况.}
\end{cases}
    \]
    不难证明$|f\uparrow f'| = |f| + |f'|$。
    \subsubsection{增广路}
    增广路是残量网络$G_f$上从$s$到$t$的一条简单路径。有了增广路$p$,很容易得到一个残量流$f_p$。
    \[
    f_p(u,v) =\begin{cases}
    c_f(p) & \text{若边$(u,v)$在路径 $p$ 上,}\\
    0 & \text{其他情况.}
\end{cases}
    \]
    其中$c_f(p) = \min\{c_f(u,v)\colon (u,v)\text{ 在路径 }p\text{ 上} \}$,$c_f(p)$称作路径$p$的残余容量。易见,$|f_p| = c_f(p) > 0$。

    增广路方法即,从图$G$上的某个初始流$f$(比如零流)开始,在$G_f$找一条增广路$p$;沿着$p$增广,更新$f$和$G_f$;如此循环,直到$G_f$上找不到增广路。此时$f$便是$G$上的一个最大流。
    \subsubsection*{流网络的割}
    为了给出最大流最小割定理,我们先介绍割的概念。
    将流网络$G=(V,E)$的点集$V$划分成两个子集$S$和$T=V-S$使得$s\in S$且$t\in T$,$(S,T)$称作$G$的一个割。令$f$为$G$上的一个流,割$(S,T)$之间的\emph{净流}$f(S,T)$定义为
    \[
    f(S,T) = \sum_{u\in S}\sum_{v\in T}f(u,v) - \sum_{u\in S}\sum_{v\in T}f(v,u)
    \]
    不难证明,对$G$的任意一个割$(S,T)$都有$f(S,T) = |f|$。
    割$(S,T)$的容量$c(S,T)$定义为
    \[
    c(S,T) = \sum_{u\in S}\sum_{v\in T}c(u,v)
    \]
    网络的最小割即所有割之中容量最小者。显然,对于$G$上的任意一个流$f$和$G$的任意一个割$(S,T)$都有 $f \le c(S,T)$。
    \begin{theorem}[最大流最小割定理]
        若$f$是流网络$G=(V,E,c,s,t)$上的一个流,则下列三个命题等价:
        \begin{enumerate}
            \item $f$是$G$上的一个最大流。
            \item 残量网络 $G_f$ 上无增广路。
            \item 存在某个割$(S,T)$满足$|f| = c(S,T)$。
        \end{enumerate}
    \end{theorem}
    \begin{proof}
        (1)$\Rightarrow$(2):显然。

        (2)$\Rightarrow$(3):假设 $G_f$中无增广路,即$G_f$上不存在从$s$到$t$的路径。令$S=\{v\in V\colon G_f\text{ 上有从}\ s\text{ 到}\ v\text{ 的路径}\}$,$T=V-S$,易见$t\notin S$,则$(S,T)$是一个割。考虑点对$u\in S$和$v\in T$。若$(u,v)\in E$,则必有$f(u,v)=c(u,v)$;因为若不然则有$(u,v)\in E_f$,即$v\in S$。若$(v,u)\in E$,则必有$f(v,u)=0$;因为若不然则有$c_f(u,v) = f(v,u) > 0$,即$(u,v)\in E_f$,仍有$v \in S$。若$(u,v)\notin E$且$(v,u)\notin E$,则$f(u,v)=f(v,u)=0$。因此我们有
        \begin{align*}
            f(S,T) &= \sum_{u\in S}\sum_{v\in T}f(u,v) - \sum_{v\in T}\sum_{u\in S}f(v,u)\\
            &= \sum_{u\in S}\sum_{v\in T}c(u,v) - \sum_{v\in T}\sum_{u\in S}0\\
            &= c(S, T)
        \end{align*}
        所以 $ |f| = f(S,T) = c(S, T)$。
        (3)$\Rightarrow$(1):由于对任意割$(S,T)$都有$|f|\le c(S,T)$,$|f|=c(S,T)$蕴含着$f$是一个最大流。
    \end{proof}

    不难看出,高效地实现增广路方法应从两个方面考虑:
    \begin{enumerate}
        \item 如何快速地在残量网络$G_f$上找一条增广路。\label{Approach:1}
        \item 如何减少增广的次数。\label{Approach:2}
    \end{enumerate}
    我们已经知道,通过深度优先搜索(DFS)或宽度优先搜索(BFS)可在线性间内找到一条增广路。
    在下一小节中我们将证明,如果每次都沿着\emph{最短增广路}(shortest augmenting path,SAP)增广,那么增广次数是$O(VE)$的。沿着最短增广路增广的算法统称为最短增广路算法。下面三个小节中要介绍的算法都属于最短增广路算法。
    \subsection{Edmonds-Karp算法}
    Edmonds-Karp是SAP算法的朴素实现。下面介绍代码实现的细节。
    
    \subsection{Dinic算法}
    \subsection{ISAP算法}
    \subsection{网络流的建图}
    \section{费用流}
    \section{二分图}
    \subsection{最大流和二分图}
    \subsection{匈牙利算法}
    \subsection{二分图模型应用}
    \section{图的连通}
    \subsection{强连通-Tarjan算法}
    \subsection{双连通}
    \subsection{2-SAT问题}
\end{document}
