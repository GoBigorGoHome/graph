\documentclass[fontset=ubuntu]{ctexbook}
\usepackage{amssymb} % \mathbb

\begin{document}
    \chapter{图论高级算法}
    \section{最大流}
    最大流算法分成两大类:增广路(augmengting path)算法与预流推进(preflow-push)算法。
    这一节介绍的三个算法,都属于增广路算法。
    下面给出几个定义。
    \subsubsection*{流网络}
    网络流的研究对象是流网络。\emph{流网络}$G=(V,E,c,s,t)$是一个有向图,$V$、$E$是其点集与边集,点和边的数目分别记作$n$、$m$。每条边(有向图的边也称作「弧」)$(u,v)\in E$都有一容量$c(u,v)\in \mathbb{N}$。$s$和$t$是网络中的两个特殊点,称作源点和汇点。为简便计,流网络简称「网络」或「图」,简记作$G=(V,E)$。

    自环在网络中无意义,我们规定图$G$中不含自环。允许网络中有重边。但是下文在论述、证明关于网络流的原理、性质或定理时,为了表示上的方便(使得两点能唯一确定一条边),认为图中不存在的重边。这个假设不妨碍一般性,我们可以通过增加一个点来消除一条重边。





    \subsection{EK算法}
    \subsection{Dinic算法}
    \subsection{ISAP算法}
    \subsection{网络流的建图}
    \section{费用流}
    \section{二分图}
    \subsection{最大流和二分图}
    \subsection{匈牙利算法}
    \subsection{二分图模型应用}
    \section{图的连通}
    \subsection{强连通-Tarjan算法}
    \subsection{双连通}
    \subsection{2-SAT问题}
\end{document}
